\chapter{\texttt{nimCSO} Tutorial} \label{chap:nimplextutorial2}}

The purpose of this guide is to demonstrate some common use cases of
\texttt{nimCSO} and go in a bit more into the details
of how it could be used, but it is not by any means extensive. If
something is not covered but you would like to see it here, please do
not hesitate to open an issue on GitHub and let use know!

\hypertarget{dataset-config-and-compilation}{%
\section{Dataset, Config, and
Compilation}\label{nimcsotutorial:dataset-config-and-compilation}}

To get started, let's first recap what we need to do to get
\texttt{nimCSO} up and running.

\textbf{1.} Install nim and dependencies, but \textbf{that's already
done for you if you are in the Codespace}. You can see what was run to
get the environment set up in the
\href{../.devcontainer/Dockerfile}{\texttt{Dockerfile}}.

\textbf{2.} Create the dataset. For now, let's just use the default one
(based on ULTERA Database) that comes with the package. Relative to this
notebook, the dataset is located at
\texttt{../dataList.txt}. Let's have a look at the
first few lines of the file to see what it looks like.

\begin{minted}[xleftmargin=3\parindent, linenos=true, fontsize=\small]{python}
!head -n 8 ../dataList.txt
\end{minted}

\begin{minted}[xleftmargin=3\parindent, fontsize=\small, bgcolor=subtlegray]{output}
Al,Co,Cr,Cu,Fe,Ni
Nb,Ta,Ti
Co,Cr,Ni
Al,Co,Cr,Fe,Mn,Ni
Al,Co,Fe,Mo,Ni
Hf,Nb,Ti,V
Co,Cr,Fe,Nb,Ni
Al,Co,Cr,Cu,Fe,Ni
\end{minted}

\textbf{3.} Now, we need to create task
\texttt{config.yaml} file that will describe what we
are doing and point to our data file. That was already done for you in
the \href{config.yaml}{\texttt{config.yaml}} file, but
you are more than welcome to play and modify it.

\textbf{4.} Finally, we can run the \texttt{nimCSO}
package to get the results. To do so, we will use one long command you
can see below. Let's break it down: - \passthrough{\lstinline"!"} is a
Jupyter Notebook magic command that allows us to run shell commands from
within the notebook.

\begin{itemize}
\item
  \texttt{nim} is the official Nim language compiler.
\item
  \texttt{c} instructs \texttt{nim}
  compiler to use \texttt{C} compiler to optimize and
  compile intermediate code. You can also use
  \texttt{cpp} to use \texttt{C++}
  compiler or \texttt{objc} to use
  \texttt{Objective-C} compiler. If you want, you can
  also compile directly with LLVM using
  \href{https://github.com/arnetheduck/nlvm}{\texttt{nlvm}},
  but it isn't pre-installed for you here.
\item
  \texttt{-f} is a flag to force the compiler to
  compile everything, even if the code didn't change. We want this
  because \texttt{config.yaml}, which tells
  \texttt{nimCSO} how to write itself, is not tracked
  by the compiler, but is critical to the compilation process (see two
  point below).
\item
  \texttt{-d:release} is a flag that tells the compiler
  to optimize the code for release. You can also use
  \texttt{-d:debug} to compile the code with better
  debugging support, but it will be slower and it will not prevent bugs
  from happening. There is also \texttt{-d:danger} that
  will disable all runtime checks and run a bit faster, but you no
  longer get memory safety guarantees.
\item
  \texttt{-d:configPath=config.yaml} is a flag pointing
  to \textbf{\texttt{config.yaml} that is read and
  tells \texttt{nimCSO} (not the compiler!) how to
  write itself \emph{before} the compilation starts.} That's the magic
  metaprogramming sauce enabling us to write functions which
  \texttt{C}/\texttt{C++} compiler can
  then turn into single deterministically allocated and exectuted
  machine code through
  \href{https://en.wikipedia.org/wiki/Inline_expansion}{inlining}.
\item
  \texttt{out:nimcso} is just telling the compiler to
  output the compiled binary right here and name it
  \texttt{nimcso}. You can name it whatever you want,
  but it's a good idea to name it something that makes sense.
\item
  \texttt{../src/nimcso} is pointing to the source code
  of \texttt{nimCSO} package to compile, relative to
  this notebook.
\end{itemize}

Let's run the command and see what happens! Shouldn't take more than a
few seconds.

\begin{minted}[xleftmargin=3\parindent, linenos=true, fontsize=\small]{python}
!nim c -f -d:release -d:configPath=config.yaml --out:nimcso ../src/nimcso 
\end{minted}

\begin{minted}[xleftmargin=3\parindent, fontsize=\small, bgcolor=subtlegray]{output}
Hint: used config file '/opt/conda/nim/config/nim.cfg' [Conf]
Hint: used config file '/opt/conda/nim/config/config.nims' [Conf]
...........................................................................................................................................................
/root/.nimble/pkgs2/nimblas-0.3.0-d5033749759fc7a2a316acf623635dcb6d69d32a/nimblas/private/common.nim(52, 7) Hint: Using BLAS library with name: lib(blas|cblas|openblas).so(||.3|.2|.1|.0) [User]
...........................................................................
config.yaml
CC: ../../../opt/conda/nim/lib/system/exceptions.nim
CC: ../../../opt/conda/nim/lib/std/private/digitsutils.nim
CC: ../../../opt/conda/nim/lib/std/assertions.nim
CC: ../../../opt/conda/nim/lib/system/dollars.nim
CC: ../../../opt/conda/nim/lib/std/syncio.nim
CC: ../../../opt/conda/nim/lib/system.nim
CC: ../../../opt/conda/nim/lib/pure/parseutils.nim
CC: ../../../opt/conda/nim/lib/pure/math.nim
CC: ../../../opt/conda/nim/lib/pure/unicode.nim
CC: ../../../opt/conda/nim/lib/pure/strutils.nim
CC: ../../../opt/conda/nim/lib/pure/hashes.nim
CC: ../../../opt/conda/nim/lib/pure/collections/sets.nim
CC: ../../../opt/conda/nim/lib/pure/times.nim
CC: ../../../opt/conda/nim/lib/std/private/ospaths2.nim
CC: ../../../opt/conda/nim/lib/std/envvars.nim
CC: ../../../opt/conda/nim/lib/std/cmdline.nim
CC: ../../../opt/conda/nim/lib/pure/collections/sequtils.nim
CC: ../../../opt/conda/nim/lib/pure/random.nim
CC: ../../../opt/conda/nim/lib/pure/collections/heapqueue.nim
CC: ../../../opt/conda/nim/lib/pure/strformat.nim
CC: ../../../opt/conda/nim/lib/pure/terminal.nim
CC: ../../../root/.nimble/pkgs2/yaml-2.1.1-302727fcd74c79d0697a4e909d26455d61a5b979/yaml/presenter.nim
CC: ../../../root/.nimble/pkgs2/arraymancer-0.7.28-d4a45ada1c7a6abebe60bcdd5ee2d7c4680799a4/arraymancer/laser/dynamic_stack_arrays.nim
CC: ../../../root/.nimble/pkgs2/arraymancer-0.7.28-d4a45ada1c7a6abebe60bcdd5ee2d7c4680799a4/arraymancer/laser/private/memory.nim
CC: ../../../root/.nimble/pkgs2/arraymancer-0.7.28-d4a45ada1c7a6abebe60bcdd5ee2d7c4680799a4/arraymancer/laser/tensor/datatypes.nim
CC: ../../../root/.nimble/pkgs2/arraymancer-0.7.28-d4a45ada1c7a6abebe60bcdd5ee2d7c4680799a4/arraymancer/laser/tensor/initialization.nim
CC: ../../../root/.nimble/pkgs2/arraymancer-0.7.28-d4a45ada1c7a6abebe60bcdd5ee2d7c4680799a4/arraymancer/tensor/init_cpu.nim
CC: ../../../root/.nimble/pkgs2/arraymancer-0.7.28-d4a45ada1c7a6abebe60bcdd5ee2d7c4680799a4/arraymancer/tensor/higher_order_applymap.nim
CC: ../../../root/.nimble/pkgs2/arraymancer-0.7.28-d4a45ada1c7a6abebe60bcdd5ee2d7c4680799a4/arraymancer/tensor/private/p_shapeshifting.nim
CC: ../../../root/.nimble/pkgs2/arraymancer-0.7.28-d4a45ada1c7a6abebe60bcdd5ee2d7c4680799a4/arraymancer/tensor/init_copy_cpu.nim
CC: ../../../root/.nimble/pkgs2/arraymancer-0.7.28-d4a45ada1c7a6abebe60bcdd5ee2d7c4680799a4/arraymancer/tensor/private/p_accessors_macros_read.nim
CC: ../../../root/.nimble/pkgs2/arraymancer-0.7.28-d4a45ada1c7a6abebe60bcdd5ee2d7c4680799a4/arraymancer/tensor/private/p_accessors_macros_write.nim
CC: ../../../root/.nimble/pkgs2/arraymancer-0.7.28-d4a45ada1c7a6abebe60bcdd5ee2d7c4680799a4/arraymancer/tensor/shapeshifting.nim
CC: ../../../root/.nimble/pkgs2/arraymancer-0.7.28-d4a45ada1c7a6abebe60bcdd5ee2d7c4680799a4/arraymancer/private/functional.nim
CC: ../../../root/.nimble/pkgs2/arraymancer-0.7.28-d4a45ada1c7a6abebe60bcdd5ee2d7c4680799a4/arraymancer/tensor/private/p_display.nim
CC: ../../../root/.nimble/pkgs2/arraymancer-0.7.28-d4a45ada1c7a6abebe60bcdd5ee2d7c4680799a4/arraymancer/tensor/display.nim
CC: ../../../root/.nimble/pkgs2/arraymancer-0.7.28-d4a45ada1c7a6abebe60bcdd5ee2d7c4680799a4/arraymancer/tensor/ufunc.nim
CC: ../../../root/.nimble/pkgs2/arraymancer-0.7.28-d4a45ada1c7a6abebe60bcdd5ee2d7c4680799a4/arraymancer/laser/cpuinfo_x86.nim
CC: ../../../root/.nimble/pkgs2/arraymancer-0.7.28-d4a45ada1c7a6abebe60bcdd5ee2d7c4680799a4/arraymancer/tensor/operators_broadcasted.nim
CC: ../../../root/.nimble/pkgs2/arraymancer-0.7.28-d4a45ada1c7a6abebe60bcdd5ee2d7c4680799a4/arraymancer/tensor/aggregate.nim
CC: nimcso/bitArrayAutoconfigured.nim
CC: nimcso.nim
Hint:  [Link]
Hint: mm: orc; threads: on; opt: speed; options: -d:release
87026 lines; 7.635s; 257.383MiB peakmem; proj: /workspaces/nimCSO/src/nimcso; out: /workspaces/nimCSO/examples/nimcso [SuccessX]
\end{minted}

Now, let's run \texttt{nimCSO} and see what happens!

\begin{minted}[xleftmargin=3\parindent, linenos=true, fontsize=\small]{python}
!./nimcso
\end{minted}

\begin{minted}[xleftmargin=3\parindent, fontsize=\small, bgcolor=subtlegray]{output}
Using 1 uint64s to store 19 elements.
Configured for task: QuickStart (Just a copy of RCCA Palette from Senkov 2018 Review)
***** nimCSO (Composition Space Optimization) *****
To use form command line, provide parameters. Currently supported usage:

--covBenchmark    | -cb   --> Run small coverage benchmarks under two implementations.
--expBenchmark    | -eb   --> Run small node expansion benchmarks.
--leastPreventing | -lp   --> Run a search for single-elements preventing the least data, i.e. the least common elements.
--mostCommon      | -mc   --> Run a search for most common elements.
--bruteForce      | -bf   --> Run brute force algorithm after getting ETA. Note that it is not feasible for more than 25ish elements.
--bruteForceInt   | -bfi  --> Run brute force algorithm with faster but not extensible uint64 representation after getting ETA. Up to 64 elements only.
--geneticSearch   | -gs   --> Run a genetic search algorithm.
--algorithmSearch | -as   --> Run a custom problem-informed best-first search algorithm.
--singleSolution  | -ss   --> Evaluate a single solution based on the elements provided as arguments after the flag. It can be stacked on itself like:
                              ./nimcso -ss Ta W Hf Si  -ss V W Hf Si  --singleSolution Ta V
\end{minted}

You should have seen a neat \texttt{help} message that
tells you how to use \texttt{nimCSO}. Let's start with
a ``coverage'' benchmark to see how fast can we check how many
datapoints will be removed from the dataset if we remove the first 5
elements of \texttt{elementOrder}.

\begin{minted}[xleftmargin=3\parindent, linenos=true, fontsize=\small]{python}
!./nimcso -cb
\end{minted}

\begin{minted}[xleftmargin=3\parindent, fontsize=\small, bgcolor=subtlegray]{output}
Using 1 uint64s to store 19 elements.
Configured for task: QuickStart (Just a copy of RCCA Palette from Senkov 2018 Review)
***** nimCSO (Composition Space Optimization) *****
Running coverage benchmark with int8 Tensor representation
Tensor[system.int8] of shape "[1, 19]" on backend "Cpu"
|1      1     1     1     1     1     0     0     0     0     0     0     0     0     0     0     0     0     0|
CPU Time [arraymancer+randomizing] 133.6μs
Prevented count:995

Running coverage benchmark with BitArray representation
CPU Time [bitty+randomizing] 13.6μs
    | 1| 2| 3| 4| 5| 6| 7| 8| 9|10|11|12|13|14|15|16|17|18|19|
 19 | 1| 1| 1| 1| 1| 1| 0| 0| 0| 0| 0| 0| 0| 0| 0| 0| 0| 0| 0|
MoTaVWTiZr->995
Prevented count:995

Running coverage benchmark with bool arrays representation (BitArray graph retained)
CPU Time [bit&boolArrays+randomizing] 16.1μs
    | 1| 2| 3| 4| 5| 6| 7| 8| 9|10|11|12|13|14|15|16|17|18|19|
 19 | 1| 1| 1| 1| 1| 1| 0| 0| 0| 0| 0| 0| 0| 0| 0| 0| 0| 0| 0|
MoTaVWTiZr->995
Prevented count:995

nimCSO Done!
\end{minted}

\hypertarget{key-routines-and-brute-forcing}{%
\section{Key Routines and Brute
Forcing}\label{nimcsotutorial:key-routines-and-brute-forcing}}

And if you were able to run that, you are all set to start using
\texttt{nimCSO}!

Let's try the simplest routine \texttt{mostCommon} or
\emph{What are the most common elements in the dataset?}

\begin{minted}[xleftmargin=3\parindent, linenos=true, fontsize=\small]{python}
!./nimcso --mostCommon
\end{minted}

\begin{minted}[xleftmargin=3\parindent, fontsize=\small, bgcolor=subtlegray]{output}
Using 1 uint64s to store 19 elements.
Configured for task: QuickStart (Just a copy of RCCA Palette from Senkov 2018 Review)
***** nimCSO (Composition Space Optimization) *****

Running search for single-elements preventing the most data.
 0: Cr->667
 1: Ti->649
 2: Fe->622
 3: Ni->620
 4: Nb->587
 5: Co->573
 6: Al->569
 7: Mo->466
 8: Zr->346
 9: Ta->330
10: V->256
11: Hf->219
12: W->207
13: Si->92
14: B->69
15: Re->55
16: C->36
17: Y->3
18: N->1

nimCSO Done!
\end{minted}

If you didn't modify anything, you should now see that elements like
\texttt{N}, \texttt{Y},
\texttt{C}, and \texttt{Re}, are not
very common in the dataset, while \texttt{Cr},
\texttt{Ti}, \texttt{Fe}, and
\texttt{Ni} are very common. When it comes to them, its
pretty obvious that removing the first group will be the first choice,
while the latter will be the last, if we want to keep the dataset as
extensive as possible.

The critical question here is, \emph{which of the intermediate elements
like \texttt{Hf}, \texttt{V},
\texttt{Ta}, or \texttt{Zr} should we
remove first?}

With a dataset spanning 19 elements, the solution space is around 0.5M,
so we can actually just brute force it in seconds :)

\begin{minted}[xleftmargin=3\parindent, linenos=true, fontsize=\small]{python}
!./nimcso -bfi
\end{minted}

\begin{minted}[xleftmargin=3\parindent, fontsize=\small, bgcolor=subtlegray]{output}
Using 1 uint64s to store 19 elements.
Configured for task: QuickStart (Just a copy of RCCA Palette from Senkov 2018 Review)
***** nimCSO (Composition Space Optimization) *****

Running brute force algorithm for 19 elements and 1349 data points.
Solution space size: 524287
Task ETA Estimate: 7 seconds and 30 milliseconds
 0: ->0
 1: N->1
 2: YN->4
 3: YCN->39
 4: ReYCN->89
 5: ReYBCN->142
 6: SiReYBCN->203
 7: WSiReYBCN->340
 8: WHfSiReYBCN->511
 9: TaWHfSiReYBCN->630
10: TaWZrHfSiReYBCN->735
11: TaVWZrHfSiReYBCN->816
12: TaVWZrHfNbSiReYBCN->859
13: TaVWTiZrHfNbSiReYBCN->952
14: MoTaVWTiZrHfNbSiReYBCN->1038
15: MoTaVWTiZrHfNbAlSiReYBCN->1304
16: TaVWTiZrHfNbCrAlCoNiReFeYCN->1327
17: MoTaVWTiZrHfNbCrAlSiCoNiReFeYB->1349
18: MoTaVWTiZrHfNbCrAlSiCoNiReFeYBC->1349
19: MoTaVWTiZrHfNbCrAlSiCoNiReFeYBCN->1349
CPU Time [Brute Force] 7308.5ms

nimCSO Done!
\end{minted}

Let's look at the result! As expected, \texttt{N},
\texttt{Y}, \texttt{C}, and
\texttt{Re} are removed first (0-4) and then the trend
follows for a bit to \texttt{Hf} \textbf{The first
break is \texttt{V}, you can notice that it's better to
remove either or both \texttt{Ta} or
\texttt{Zr} first, despite the fact that they are
nearly 50\% more common than \texttt{V}} That's
because they often coocur with \texttt{Re} and
\texttt{Hf}, which are not common.

We can test exactly how much more data we will have if we remove
\texttt{Ta} insead of \texttt{V} by
using the \texttt{--singleSolution} /
\texttt{-ss} routine.

\begin{minted}[xleftmargin=3\parindent, linenos=true, fontsize=\small]{python}
!./nimcso -ss Ta W Hf Si Re Y B C N -ss V W Hf Si Re Y B C N
\end{minted}

\begin{minted}[xleftmargin=3\parindent, fontsize=\small, bgcolor=subtlegray]{output}
Using 1 uint64s to store 19 elements.
Configured for task: QuickStart (Just a copy of RCCA Palette from Senkov 2018 Review)
***** nimCSO (Composition Space Optimization) *****
Testing solution with @[@["Ta", "W", "Hf", "Si", "Re", "Y", "B", "C", "N"], @["V", "W", "Hf", "Si", "Re", "Y", "B", "C", "N"]]
 9: TaWHfSiReYBCN->630
 9: VWHfSiReYBCN->697

nimCSO Done!
\end{minted}

Wow! Looking at the \texttt{--mostCommon} output from
earlier, we can see that \textbf{\texttt{Ta} is present
in 74 more datapoints than \texttt{V}, but after
removing \texttt{WHfSiReYBCN}, picking
\texttt{V} as one of 10 elements to model will result
in 67 \emph{more} datapoints.} Relative to a dataset without
interdependencies, that's a 141 datapoint difference!

And another case that breaks from the ordering is
\texttt{Mo}, which is better to keep than much more
common \texttt{Nb}, and after
\texttt{Nb} is removed, even better thank keeping the
\texttt{Ti}, which is the second most common element in
the dataset!

Similarly to what we did with \texttt{V}
vs.~\texttt{Ta}, we can test how much more data we will
have if we remove \texttt{Nb} instead of
\texttt{Mo} by using the
\texttt{--singleSolution} /
\texttt{-ss} routine.

\begin{minted}[xleftmargin=3\parindent, linenos=true, fontsize=\small]{python}
!./nimcso -ss Ta V W Zr Hf Nb Si Re Y B C N -ss Ta V W Zr Hf Mo Si Re Y B C N -ss Ta V W Zr Hf Ti Si Re Y B C N
\end{minted}

\begin{minted}[xleftmargin=3\parindent, fontsize=\small, bgcolor=subtlegray]{output}
Using 1 uint64s to store 19 elements.
Configured for task: QuickStart (Just a copy of RCCA Palette from Senkov 2018 Review)
***** nimCSO (Composition Space Optimization) *****
Testing solution with @[@["Ta", "V", "W", "Zr", "Hf", "Nb", "Si", "Re", "Y", "B", "C", "N"], @["Ta", "V", "W", "Zr", "Hf", "Mo", "Si", "Re", "Y", "B", "C", "N"], @["Ta", "V", "W", "Zr", "Hf", "Ti", "Si", "Re", "Y", "B", "C", "N"]]
12: TaVWZrHfNbSiReYBCN->859
12: MoTaVWZrHfSiReYBCN->935
12: TaVWTiZrHfSiReYBCN->938

nimCSO Done!
\end{minted}

We can see that \textbf{\texttt{Nb} is present in 121
more datapoints than \texttt{Mo}, but after removing
\texttt{TaVWZrHfSiReYBCN}, picking
\texttt{Mo} as one of 7 elements to model will result
in 76 \emph{more} datapoints.} Relative to a dataset without
interdependencies, that's a 197 datapoint difference, which is even more
than the \texttt{Ta} vs.~\texttt{V}
case! Additionally, we can see that \texttt{Ti} is only
3 datapoints better than \texttt{Mo}, despite being
present in 183 more datapoints than \texttt{Mo}.

\hypertarget{algorithm-search}{%
\section{Algorithm Search}\label{nimcsotutorial:algorithm-search}}

The
\texttt{--bruteForceInt}/\texttt{-bfi}
routine we used to find the solutions worked great for our 19-element
dataset and took only a few seconds on the low-performance Codespace
machine, but in many cases dimensionality of the problem will be too
high to brute force it.

Let's now try to use the
\texttt{--algorithmSearch}/\texttt{-as}
routine, which takes advantage of some assumptions known to be valid or
likely to be valid (see manuscript), to limit the search space and find
the solution in a reasonable time. Let's try it now!

\begin{minted}[xleftmargin=3\parindent, linenos=true, fontsize=\small]{python}
!./nimcso -as
\end{minted}

\begin{minted}[xleftmargin=3\parindent, fontsize=\small, bgcolor=subtlegray]{output}
Using 1 uint64s to store 19 elements.
Configured for task: QuickStart (Just a copy of RCCA Palette from Senkov 2018 Review)
***** nimCSO (Composition Space Optimization) *****

Running Algorithm Search for 19 elements.
 1: N->1  (tree size: 19)
 2: YN->4  (tree size: 52)
 3: YCN->39  (tree size: 113)
 4: ReYCN->89  (tree size: 275)
 5: ReYBCN->142  (tree size: 581)
 6: SiReYBCN->203  (tree size: 690)
 7: WSiReYBCN->340  (tree size: 1818)
 8: WHfSiReYBCN->511  (tree size: 3873)
 9: TaWHfSiReYBCN->630  (tree size: 5213)
10: TaWZrHfSiReYBCN->735  (tree size: 4833)
11: TaVWZrHfSiReYBCN->816  (tree size: 4192)
12: TaVWZrHfNbSiReYBCN->859  (tree size: 3784)
13: TaVWTiZrHfNbSiReYBCN->952  (tree size: 2955)
14: MoTaVWTiZrHfNbSiReYBCN->1038  (tree size: 1765)
15: MoTaVWTiZrHfNbAlSiReYBCN->1304  (tree size: 45)
16: MoTaVWTiZrHfNbAlSiReFeYBCN->1338  (tree size: 8)
17: MoTaVWTiZrHfNbCrAlSiReFeYBCN->1349  (tree size: 4)
18: MoTaVWTiZrHfNbCrAlSiNiReFeYBCN->1349  (tree size: 4)
CPU Time [exploring] 109.7ms

nimCSO Done!
\end{minted}

As you can see, \textbf{the algorithm reproduced the same results as the
brute force search around 100 times faster}, except for third-to-last
step because dataset had points with at least 3 elements breaking its
backtracking assumptions.

\hypertarget{genetic-search}{%
\section{Genetic Search}\label{nimcsotutorial:genetic-search}}

For cases where the dimensionality of the problem is too high to either
brute-force or use the algorithm search, we can still use the
\texttt{--geneticSearch}/\texttt{-gs}
routine to find the solution in a reasonable time. Let's try it now!

Please note that the results are stochastic, so you might get different
results than ones shown below if you run the command again.

\begin{minted}[xleftmargin=3\parindent, linenos=true, fontsize=\small]{python}
!./nimcso -gs
\end{minted}

\begin{minted}[xleftmargin=3\parindent, fontsize=\small, bgcolor=subtlegray]{output}
Using 1 uint64s to store 19 elements.
Configured for task: QuickStart (Just a copy of RCCA Palette from Senkov 2018 Review)
***** nimCSO (Composition Space Optimization) *****

Running Genetic Search algorithm for 19 elements and 1349 data points.
Initiating each level with 100 random solutions and expanding 100 solutions at each level for up to 1000 iterations.
 1: N->1
 2: YN->4  (queue size: 256)
 3: YCN->39  (queue size: 615)
 4: ReYCN->89  (queue size: 869)
 5: ReYBCN->142  (queue size: 929)
 6: SiReYBCN->203  (queue size: 1379)
 7: WSiReYBCN->340  (queue size: 1267)
 8: WHfSiReYBCN->511  (queue size: 1631)
 9: TaWHfSiReYBCN->630  (queue size: 1578)
10: TaWZrHfSiReYBCN->735  (queue size: 1835)
11: TaVWZrHfSiReYBCN->816  (queue size: 1621)
12: TaVWZrHfNbSiReYBCN->859  (queue size: 1746)
13: VWCrAlSiCoNiReFeYBCN->1176  (queue size: 1713)
14: MoTaVWTiZrHfNbSiReYBCN->1038  (queue size: 1565)
15: MoTaVWCrAlSiCoNiReFeYBCN->1320  (queue size: 1028)
16: TaVWTiZrHfNbCrAlCoNiReFeYCN->1327  (queue size: 1575)
17: MoTaVWTiZrHfNbCrAlSiCoFeYBCN->1349  (queue size: 268)
18: MoTaVWTiZrHfNbCrAlSiCoNiReFeBCN->1349  (queue size: 18)
CPU Time [Genetic Search] 766.9ms

nimCSO Done!
\end{minted}

\hypertarget{summary}{%
\subsection{Summary}\label{nimcsotutorial:summary}}

Now, you should be able to apply \texttt{nimCSO} to
your own dataset and get some valuable insights on how to model it!

If you are working in a Codespace, you can just do everything right in
this notebook by simply modifying the
\texttt{config.yaml} file and running the commands you
just learned about. The Codespace will be persisted until you explicitly
delete it, so you can come back to it later and continue your work by
clicking on the link in the ``Open in Codespaces'' badge in the README
of the repository and resuming your work.
