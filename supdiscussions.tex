\chapter{Supplementary Discussions} \label{chap:supdiscussions}

\section{Intermediate \texttt{SIPFENN} Models} \label{sipfenn:appendix2}

The neural network design process was conducted in incremental fashion, starting from a perceptron, which is the simplest type of neural network proposed by Frank Rosenblatt in 1957 \cite{Rosenblatt1957TheAutomaton}. It effectively operates as a linear function $f(\vec{d}) = A(w_1 d_1 + w_2 d_2 + ... + w_n d_n)$ where $d_i$ is i-th element of the descriptor $\vec{d}$, $w_i$ is the weight associated with it, and $A$ is an activation function that can introduce non-linearity or turn it into a classifier. Here, the popular Sigmoid activation function was used. 

The perceptron was first trained on the data from the first 5000 entries in the ICSD, to check whether the training was set up correctly. It achieved a MAE of 195 meV/atom on the test set of 230 randomly selected entries ($\approx 5\% \text{ from } 5000$). Results are shown in Figure \ref{sipfenn:fig:nn1performance}. 

\begin{figure}[H]
    \centering
    \includegraphics[width=0.3\textwidth]{sipfenn/NN1_test.png}
    \caption{Test of perceptron trained on the data from the first 5000 entries in the ICSD dataset and evaluated on the test set of 230 randomly selected entries ($\approx5\%$)}
    \label{sipfenn:fig:nn1performance}
\end{figure}

When trained on the data from all entries in the ICSD, it achieved an MAE of 364 meV/atom on the test set ($\approx5\% \text{ from } 32116$). This error is comparable to the performance of a random forest model based on PRDF (370 meV/atom), is slightly worse than a CM (250 meV/atom), and is significantly worse than a random-forest model trained on the same descriptor (90 meV/atom), as reported by Ward et al. \cite{Ward2017IncludingTessellations}. Part of the significance of these results is the evident quality of the descriptor, as the model achieved performance that would be considered excellent just a few years prior to the present work while being much less complex and computationally costly. Furthermore, it is important to note the time- and space-complexity of the perceptron model. Training the final network took less than 8 seconds compared to around 10,000 seconds reported for the aforementioned random-forest methods, and the resulting model occupied less than 1kb of memory. Following the testing of a perceptron, which allowed rough estimation of the a good size of the network (i.e. number of weights), the design of the actual architecture began. All of these steps are schematically depicted in Figure \ref{sipfenn:fig:designprocess}.

\begin{figure}[H]
    \centering
    \includegraphics[width=0.4\textwidth]{sipfenn/SIPFENN_design_updated.png}
    \caption{The network design process schematic leading to the three final models. This figure is Figure \ref{sipfenn:fig:designprocess} cloned to the Appendix.}
\end{figure}

Next, in a few steps, the size of the network was incrementally increased. First, a layer of 1000 neurons was introduced. This reduced the performance on the first 5000 entries in the ICSD, likely due to overfitting issues, as the data was very limited. Performance on the ICSD was improved, reducing the test MAE to 305 meV/atom on the test set, however. The introduction of the next two 1000-width layers further reduced the MAE to 215 meV/atom. Based on these results, it was estimated that introducing 4 hidden layers with Sigmoid activation function and widths of 10000, 10000, 1000, and 100 would provide good results when trained on the much larger OQMD.

After switching to OQMD, the network exhibited issues with convergence, often predicting a single value for all of the entries. To mitigate this, the descriptor (i.e. network input) was normalized by dividing every element by its maximum value across the whole dataset. This solved the issue. Next, to improve the training behavior, the activation functions were changed from only the Sigmoid function to a mix of Soft Sign, Exponential Linear Unit, and Sigmoid, which was found to work well. These steps improved both the predictive performance and reduced the time required to converge. The network architecture resulting from these steps (internally designated NN8 / Simple Good Network in Figure \ref{sipfenn:fig:designprocess}) was the first to improve performance compared to the Ward et. al approach \cite{Ward2017IncludingTessellations}, achieving an MAE of 42 meV/atom on the test set of random subset 5\% of the OQMD dataset. When testing this network, a small fraction of around 0.03\% of likely incorrect entries in the OQMD was found, as described in \ref{sipfenn:sssec:Data}, and was removed from the dataset used later in the design process.

Once a network with desired performance was obtained, the network size was increased until it either exceeded 1GB or showed signs of extensive overfitting. At the first step of this process, two layers of width 10,000 were added, resulting in a network size of 1.2GB and reduced overfitting, as indicated by the ratio of validation-to-training error lowered from 2.2 to 1.6, relative to NN8. The resulting network (internally designated NN9 / OQMD-Optimized Network in Figure \ref{sipfenn:fig:designprocess}), achieved an MAE of 28 meV/atom on the test set of random subset 5\% of OQMD, which was the best performance on OQMD out of all the networks created in this project. If the 0.03\% of abnormal data wasn't removed as described in \ref{sipfenn:sssec:Data}, it would correspond to, on average, 6 data points which in one tested instance increased the MAE to 35 meV/atom. Important to point out, the training of this network was prone to staying in local minima at the beginning. The reported instance of the trained network exhibited no training progress between around rounds 5 and 25, after which it's performance quickly increased.  Detailed analysis of the performance is given in \ref{sipfenn:ssec:oqmdperformance}.

Once the main objective of the design process was obtained, i.e. the performance on the OQMD has improved appreciably beyond existing methods, the design process was focused on creating a tool for modeling materials that were not reported in the OQMD. Therefore, the objective changed from achieving the lowest MAE on a random subset 5\% of OQMD to (1) reducing the mismatch between training and validation sets errors (i.e. difference between training accuracy and validation accuracy) during the training process, (2) keeping the test MAE on the OQMD below 50 meV/atom, and (3) improving performance on two material groups significantly different from the OQMD data, namely Special Quasirandom Structures (SQS) and Fe-Cr-Ni $\sigma$-phase (see \ref{sipfenn:sssec:Data}).

With these new objectives, two Dropout layers in the middle part of the network were introduced to promote the distribution of pattern recognition abilities across the network. \cite{Srivastava2014Dropout:Overfitting} This introduced a problem with convergence as the network became more likely to fall into local minima at the initial stages of the training, which was solved by introducing custom learning rate schedules. Specifically, the learning rate was initially set to a value orders of magnitude lower than during the default initial training and then ramped up to the previous (ADAM default setting in the majority of frameworks) learning rate of 0.001 (or above) after around 2 rounds of training. This type of learning rate schedule is known as a warm-up in the deep learning literature \cite{gotmare2018closer}. The schedule found to perform the best is presented in Figure \ref{sipfenn:fig:learningrate}.

\begin{figure}[H]
\centering
    \includegraphics[width=0.35\textwidth]{sipfenn/nn18_learningrate_linonly.png}
    \caption{The learning rate schedule used for training of more complex networks in the later stage of the design process (e.g., NN18).}
    \label{sipfenn:fig:learningrate}
\end{figure}


The next step was the introduction of $\ell^2$ regularization, which is a technique that favors simplification of the descriptor and effectively rejects features of the descriptor that do not contribute to prediction performance \cite{L2Regularization}. An overview on it is given in Section \ref{sipfenn:ref:machinelearningoverview}. In the models reported in the present work an $\ell^2$ value of $10^{-6}$ was used. Higher values were found to stop the training at early stages, impairing the pattern recognition, or in extreme cases (above $10^{-3}$) force the network to discard the input completely, resulting in constant or near-constant output (i.e. mean value from the training dataset predicted for any structure).

The final step was small curation of the training data based on the OQMD-reported structure stability, i.e. the energy difference between the formation energy and the energy convex hull. The motivation for that was the notion that DFT results are inherently less accurate for unstable phases. In this step, all entries with energies of more than 2000 meV/atom above the convex hull were removed from the training set. Importantly, the validation and testing sets were not modified for consistent performance reporting.

All of these changes resulted in a neural network that has been optimized for predicting new materials. In the code and Supplementary materials, it is designated as NN20 (Novel Materials Network in Figure \ref{sipfenn:fig:designprocess}). Compared to the OQMD-optimized network it was derived from, the test MAE on the OQMD increased from 28 to 49 meV/atom. However, at the same time, the mismatch between the training and validation set was reduced from 1.57 to 1.38. Or, as presented earlier in Figure \ref{sipfenn:fig:trainingvalidation}, reduced to about 1.15 for the same training duration. Furthermore, a relatively large portion of this error can be attributed to some unstable structures that were removed from the training set, but not from the test set. Once entries with formation energies of more than 1000 meV/atom above the convex hull were removed, the test MAE decreased to only 38 meV/atom. Restricting the test set further to only somewhat stable structures (stability below 250 meV/atom) resulted in an MAE of 30 meV/atom.

While the new-material-optimized network presented an increased MAE across a random subset of the OQMD, performance has significantly improved on the Fe-Cr-Ni $\sigma-$phase described in \ref{sipfenn:sssec:Data}. The MAE has decreased from 55 to 41 meV/atom, indicating that the model based on this neural network is more capable of making predictions for new materials.

Once two performance-oriented models were developed, increasing the performance-to-cost ratio has been explored with the motivation that some studies would benefit from many times higher throughput at minor accuracy decrease. Architecture design started from the selection of a network with a balanced size-to-performance ratio (NN8) and the introduction of an overfitting mitigation technique (Dropout \cite{srivastava2014dropout}) used for the network optimized for new materials, as depicted in Figure \ref{sipfenn:sssec:NetDesign}. Next, the network was gradually narrowed (fewer neurons in layers) until the performance started to noticeably deteriorate (41.9 meV/atom for 5000- and 4000-width vs 42.1 for 3000-width). This approach allowed a significant reduction of the network size (and the computational intensity to run it) from around 1,200MB of the two other models to around 145MB. If an application demands even more of a reduction in model size and computational cost, the same procedure could be continued until some minimum required performance is retained. 


\section{Feature Ranking Learned During Formation Energy Modeling} 
\label{sipfenn:appendix3}

\begin{longtable}{|l|l|}
\caption{\texttt{SIPFENN}'s \texttt{NN20} Model Input Feature Ranking Learned During Formation Energy Modeling}
\label{sipfenn:appendix3:featureranking}\\
\hline
\multicolumn{1}{|c|}{\textbf{Descriptor Feature}} & \multicolumn{1}{c|}{\textbf{Normalized Squared Weights Sum}} \\ \hline
\endfirsthead
%
\endhead
%
mean\_NeighDiff\_shell1\_MeltingT & 1 \\ \hline
mean\_MeltingT & 0.97502 \\ \hline
max\_MeltingT & 0.73512 \\ \hline
mean\_NeighDiff\_shell1\_NdUnfilled & 0.69157 \\ \hline
MaxPackingEfficiency & 0.68889 \\ \hline
most\_MeltingT & 0.67373 \\ \hline
dev\_GSvolume\_pa & 0.61042 \\ \hline
var\_NeighDiff\_shell1\_Column & 0.58782 \\ \hline
var\_NeighDiff\_shell1\_CovalentRadius & 0.57826 \\ \hline
var\_NeighDiff\_shell1\_MeltingT & 0.57259 \\ \hline
maxdiff\_GSvolume\_pa & 0.55156 \\ \hline
dev\_MeltingT & 0.5286 \\ \hline
mean\_SpaceGroupNumber & 0.51761 \\ \hline
min\_MeltingT & 0.50437 \\ \hline
var\_CellVolume & 0.49467 \\ \hline
var\_NeighDiff\_shell1\_MendeleevNumber & 0.492 \\ \hline
min\_NeighDiff\_shell1\_MeltingT & 0.47853 \\ \hline
mean\_NeighDiff\_shell1\_Column & 0.45566 \\ \hline
maxdiff\_CovalentRadius & 0.42998 \\ \hline
var\_NeighDiff\_shell1\_Electronegativity & 0.42642 \\ \hline
var\_EffectiveCoordination & 0.40506 \\ \hline
min\_NeighDiff\_shell1\_Column & 0.39822 \\ \hline
dev\_NdUnfilled & 0.39739 \\ \hline
dev\_CovalentRadius & 0.36935 \\ \hline
range\_NeighDiff\_shell1\_Column & 0.35956 \\ \hline
range\_NeighDiff\_shell1\_CovalentRadius & 0.34585 \\ \hline
mean\_WCMagnitude\_Shell1 & 0.34275 \\ \hline
mean\_NeighDiff\_shell1\_MendeleevNumber & 0.33911 \\ \hline
mean\_EffectiveCoordination & 0.33899 \\ \hline
mean\_Number & 0.33769 \\ \hline
mean\_NdUnfilled & 0.33408 \\ \hline
maxdiff\_MeltingT & 0.33348 \\ \hline
mean\_AtomicWeight & 0.33149 \\ \hline
mean\_NeighDiff\_shell1\_NdValence & 0.33142 \\ \hline
range\_NeighDiff\_shell1\_MeltingT & 0.33107 \\ \hline
max\_NfUnfilled & 0.33041 \\ \hline
dev\_Electronegativity & 0.33001 \\ \hline
mean\_NeighDiff\_shell1\_CovalentRadius & 0.32999 \\ \hline
var\_NeighDiff\_shell1\_NdUnfilled & 0.31973 \\ \hline
dev\_Column & 0.31662 \\ \hline
var\_NeighDiff\_shell1\_NdValence & 0.31481 \\ \hline
mean\_WCMagnitude\_Shell2 & 0.31359 \\ \hline
most\_NfUnfilled & 0.30916 \\ \hline
MeanIonicChar & 0.30732 \\ \hline
mean\_NeighDiff\_shell1\_Electronegativity & 0.30277 \\ \hline
min\_EffectiveCoordination & 0.29705 \\ \hline
min\_NeighDiff\_shell1\_CovalentRadius & 0.29392 \\ \hline
max\_NeighDiff\_shell1\_GSvolume\_pa & 0.2875 \\ \hline
most\_SpaceGroupNumber & 0.28472 \\ \hline
max\_NdUnfilled & 0.28424 \\ \hline
maxdiff\_NdUnfilled & 0.28405 \\ \hline
var\_NeighDiff\_shell1\_GSvolume\_pa & 0.28008 \\ \hline
min\_BondLengthVariation & 0.27922 \\ \hline
var\_MeanBondLength & 0.2768 \\ \hline
dev\_NdValence & 0.27566 \\ \hline
max\_NeighDiff\_shell1\_MeltingT & 0.27097 \\ \hline
max\_BondLengthVariation & 0.26565 \\ \hline
mean\_NfValence & 0.26558 \\ \hline
mean\_NsUnfilled & 0.2612 \\ \hline
max\_NeighDiff\_shell1\_CovalentRadius & 0.26026 \\ \hline
max\_GSvolume\_pa & 0.25985 \\ \hline
min\_GSvolume\_pa & 0.25895 \\ \hline
mean\_NdValence & 0.25573 \\ \hline
mean\_NeighDiff\_shell1\_GSvolume\_pa & 0.25299 \\ \hline
max\_NValance & 0.24749 \\ \hline
range\_NeighDiff\_shell1\_NdUnfilled & 0.24643 \\ \hline
max\_CovalentRadius & 0.23136 \\ \hline
CanFormIonic & 0.23135 \\ \hline
min\_NeighDiff\_shell1\_Electronegativity & 0.22873 \\ \hline
min\_SpaceGroupNumber & 0.22766 \\ \hline
max\_Electronegativity & 0.22609 \\ \hline
max\_NdValence & 0.22576 \\ \hline
most\_NdUnfilled & 0.22198 \\ \hline
min\_NeighDiff\_shell1\_MendeleevNumber & 0.21991 \\ \hline
var\_NeighDiff\_shell1\_NpValence & 0.21609 \\ \hline
min\_NeighDiff\_shell1\_NdUnfilled & 0.2114 \\ \hline
dev\_SpaceGroupNumber & 0.2099 \\ \hline
most\_NfValence & 0.20888 \\ \hline
min\_MeanBondLength & 0.2086 \\ \hline
mean\_BondLengthVariation & 0.20507 \\ \hline
var\_NeighDiff\_shell1\_Row & 0.20454 \\ \hline
max\_NeighDiff\_shell1\_NdUnfilled & 0.20318 \\ \hline
min\_NeighDiff\_shell1\_NdValence & 0.20123 \\ \hline
min\_CovalentRadius & 0.19974 \\ \hline
range\_NeighDiff\_shell1\_MendeleevNumber & 0.19591 \\ \hline
min\_NeighDiff\_shell1\_GSvolume\_pa & 0.19565 \\ \hline
most\_NpUnfilled & 0.19457 \\ \hline
maxdiff\_NUnfilled & 0.19316 \\ \hline
max\_NeighDiff\_shell1\_NdValence & 0.19307 \\ \hline
max\_NpValence & 0.1929 \\ \hline
range\_NeighDiff\_shell1\_GSvolume\_pa & 0.19166 \\ \hline
most\_NdValence & 0.1904 \\ \hline
max\_MeanBondLength & 0.19021 \\ \hline
maxdiff\_NfUnfilled & 0.18897 \\ \hline
max\_NeighDiff\_shell1\_Column & 0.18518 \\ \hline
range\_NeighDiff\_shell1\_Electronegativity & 0.18322 \\ \hline
var\_NeighDiff\_shell1\_SpaceGroupNumber & 0.18313 \\ \hline
dev\_NpValence & 0.18099 \\ \hline
mean\_NpUnfilled & 0.18091 \\ \hline
range\_NeighDiff\_shell1\_SpaceGroupNumber & 0.17858 \\ \hline
dev\_MendeleevNumber & 0.17753 \\ \hline
MaxIonicChar & 0.176 \\ \hline
mean\_Column & 0.17206 \\ \hline
min\_Electronegativity & 0.17164 \\ \hline
mean\_WCMagnitude\_Shell3 & 0.17077 \\ \hline
mean\_Row & 0.17035 \\ \hline
min\_NeighDiff\_shell1\_SpaceGroupNumber & 0.17031 \\ \hline
most\_NsUnfilled & 0.16714 \\ \hline
var\_BondLengthVariation & 0.16653 \\ \hline
var\_NeighDiff\_shell1\_NfUnfilled & 0.16223 \\ \hline
range\_NeighDiff\_shell1\_NdValence & 0.16094 \\ \hline
frac\_fValence & 0.1609 \\ \hline
maxdiff\_Column & 0.16083 \\ \hline
max\_NUnfilled & 0.15916 \\ \hline
mean\_NpValence & 0.15639 \\ \hline
maxdiff\_NpValence & 0.15637 \\ \hline
mean\_MendeleevNumber & 0.15491 \\ \hline
most\_Electronegativity & 0.15469 \\ \hline
mean\_Electronegativity & 0.15458 \\ \hline
max\_SpaceGroupNumber & 0.15429 \\ \hline
dev\_Row & 0.15382 \\ \hline
maxdiff\_MendeleevNumber & 0.15373 \\ \hline
var\_NeighDiff\_shell1\_NpUnfilled & 0.15135 \\ \hline
max\_NeighDiff\_shell1\_Electronegativity & 0.15115 \\ \hline
most\_NUnfilled & 0.14955 \\ \hline
max\_GSbandgap & 0.14945 \\ \hline
mean\_NeighDiff\_shell1\_NUnfilled & 0.14891 \\ \hline
maxdiff\_NValance & 0.14819 \\ \hline
mean\_NeighDiff\_shell1\_NpValence & 0.14768 \\ \hline
maxdiff\_NdValence & 0.14735 \\ \hline
max\_NpUnfilled & 0.14647 \\ \hline
maxdiff\_Electronegativity & 0.14523 \\ \hline
min\_MendeleevNumber & 0.14119 \\ \hline
mean\_CovalentRadius & 0.14049 \\ \hline
mean\_NeighDiff\_shell1\_Row & 0.13945 \\ \hline
maxdiff\_GSbandgap & 0.13891 \\ \hline
max\_NeighDiff\_shell1\_MendeleevNumber & 0.13858 \\ \hline
most\_Number & 0.13823 \\ \hline
most\_AtomicWeight & 0.13798 \\ \hline
max\_NeighDiff\_shell1\_NpValence & 0.13757 \\ \hline
Comp\_L10Norm & 0.13598 \\ \hline
min\_Row & 0.13596 \\ \hline
range\_NeighDiff\_shell1\_NpValence & 0.13524 \\ \hline
mean\_GSvolume\_pa & 0.1331 \\ \hline
max\_NeighDiff\_shell1\_NUnfilled & 0.13205 \\ \hline
mean\_NeighDiff\_shell1\_NfValence & 0.12888 \\ \hline
min\_NeighDiff\_shell1\_NpUnfilled & 0.12778 \\ \hline
mean\_NeighDiff\_shell1\_SpaceGroupNumber & 0.12722 \\ \hline
mean\_NsValence & 0.12642 \\ \hline
most\_CovalentRadius & 0.12616 \\ \hline
var\_NeighDiff\_shell1\_NUnfilled & 0.12525 \\ \hline
mean\_NeighDiff\_shell1\_Number & 0.12466 \\ \hline
Comp\_L7Norm & 0.12293 \\ \hline
mean\_NeighDiff\_shell1\_AtomicWeight & 0.12229 \\ \hline
min\_NeighDiff\_shell1\_NpValence & 0.12026 \\ \hline
max\_EffectiveCoordination & 0.11995 \\ \hline
min\_NdValence & 0.11984 \\ \hline
maxdiff\_NpUnfilled & 0.11976 \\ \hline
mean\_NeighDiff\_shell1\_NsUnfilled & 0.11836 \\ \hline
max\_NeighDiff\_shell1\_GSbandgap & 0.11657 \\ \hline
min\_NUnfilled & 0.11648 \\ \hline
most\_Column & 0.1164 \\ \hline
var\_NeighDiff\_shell1\_Number & 0.11483 \\ \hline
most\_MendeleevNumber & 0.11312 \\ \hline
max\_NeighDiff\_shell1\_SpaceGroupNumber & 0.11292 \\ \hline
var\_NeighDiff\_shell1\_AtomicWeight & 0.11234 \\ \hline
most\_NpValence & 0.11231 \\ \hline
frac\_dValence & 0.11126 \\ \hline
NComp & 0.11097 \\ \hline
min\_Number & 0.11062 \\ \hline
range\_NeighDiff\_shell1\_NpUnfilled & 0.11002 \\ \hline
dev\_NValance & 0.10868 \\ \hline
min\_Column & 0.10846 \\ \hline
max\_NeighDiff\_shell1\_NpUnfilled & 0.10837 \\ \hline
maxdiff\_Row & 0.10735 \\ \hline
Comp\_L5Norm & 0.10726 \\ \hline
mean\_NeighDiff\_shell1\_NpUnfilled & 0.10682 \\ \hline
maxdiff\_SpaceGroupNumber & 0.10604 \\ \hline
dev\_GSbandgap & 0.10604 \\ \hline
max\_AtomicWeight & 0.10495 \\ \hline
max\_GSmagmom & 0.10416 \\ \hline
maxdiff\_GSmagmom & 0.1039 \\ \hline
dev\_NUnfilled & 0.10336 \\ \hline
var\_NeighDiff\_shell1\_NfValence & 0.10059 \\ \hline
dev\_GSmagmom & 0.10046 \\ \hline
most\_GSbandgap & 0.09997 \\ \hline
var\_NeighDiff\_shell1\_NValance & 0.09842 \\ \hline
min\_NeighDiff\_shell1\_Row & 0.09798 \\ \hline
min\_NeighDiff\_shell1\_NUnfilled & 0.09563 \\ \hline
most\_Row & 0.09538 \\ \hline
max\_Number & 0.0925 \\ \hline
most\_GSvolume\_pa & 0.09166 \\ \hline
mean\_GSbandgap & 0.09097 \\ \hline
range\_NeighDiff\_shell1\_Row & 0.09081 \\ \hline
mean\_NValance & 0.0889 \\ \hline
mean\_NeighDiff\_shell1\_NsValence & 0.08449 \\ \hline
min\_NsValence & 0.08408 \\ \hline
frac\_pValence & 0.08403 \\ \hline
mean\_NUnfilled & 0.08244 \\ \hline
mean\_NfUnfilled & 0.08194 \\ \hline
dev\_NpUnfilled & 0.0818 \\ \hline
dev\_Number & 0.08065 \\ \hline
max\_NeighDiff\_shell1\_GSmagmom & 0.08049 \\ \hline
max\_Column & 0.07989 \\ \hline
min\_AtomicWeight & 0.07959 \\ \hline
Comp\_L3Norm & 0.07913 \\ \hline
max\_NeighDiff\_shell1\_Row & 0.0776 \\ \hline
mean\_NeighDiff\_shell1\_NValance & 0.07619 \\ \hline
mean\_NeighDiff\_shell1\_NfUnfilled & 0.07413 \\ \hline
range\_NeighDiff\_shell1\_NfUnfilled & 0.07381 \\ \hline
min\_NValance & 0.07297 \\ \hline
max\_NeighDiff\_shell1\_NValance & 0.0726 \\ \hline
range\_NeighDiff\_shell1\_NfValence & 0.07163 \\ \hline
min\_NdUnfilled & 0.07145 \\ \hline
most\_NsValence & 0.07114 \\ \hline
mean\_NeighDiff\_shell1\_GSbandgap & 0.06709 \\ \hline
max\_NfValence & 0.06661 \\ \hline
dev\_AtomicWeight & 0.06581 \\ \hline
maxdiff\_Number & 0.06576 \\ \hline
max\_NeighDiff\_shell1\_NfUnfilled & 0.06523 \\ \hline
dev\_NfUnfilled & 0.06477 \\ \hline
dev\_NfValence & 0.06373 \\ \hline
range\_NeighDiff\_shell1\_GSmagmom & 0.06305 \\ \hline
var\_NeighDiff\_shell1\_NsUnfilled & 0.06288 \\ \hline
min\_NeighDiff\_shell1\_Number & 0.0623 \\ \hline
frac\_sValence & 0.06099 \\ \hline
min\_NeighDiff\_shell1\_NfValence & 0.06033 \\ \hline
max\_Row & 0.05998 \\ \hline
min\_NeighDiff\_shell1\_NValance & 0.05844 \\ \hline
range\_NeighDiff\_shell1\_NUnfilled & 0.05819 \\ \hline
var\_NeighDiff\_shell1\_GSbandgap & 0.05683 \\ \hline
range\_NeighDiff\_shell1\_AtomicWeight & 0.0568 \\ \hline
Comp\_L2Norm & 0.05638 \\ \hline
min\_NeighDiff\_shell1\_NsUnfilled & 0.05541 \\ \hline
most\_NValance & 0.0553 \\ \hline
maxdiff\_NsValence & 0.05459 \\ \hline
range\_NeighDiff\_shell1\_NValance & 0.0537 \\ \hline
min\_NeighDiff\_shell1\_AtomicWeight & 0.05369 \\ \hline
max\_NsValence & 0.05329 \\ \hline
range\_NeighDiff\_shell1\_GSbandgap & 0.05299 \\ \hline
min\_NeighDiff\_shell1\_NfUnfilled & 0.05266 \\ \hline
maxdiff\_NfValence & 0.05147 \\ \hline
dev\_NsUnfilled & 0.04884 \\ \hline
max\_MendeleevNumber & 0.04844 \\ \hline
maxdiff\_AtomicWeight & 0.04814 \\ \hline
max\_NeighDiff\_shell1\_NsUnfilled & 0.04675 \\ \hline
max\_NeighDiff\_shell1\_NsValence & 0.04663 \\ \hline
var\_NeighDiff\_shell1\_GSmagmom & 0.04635 \\ \hline
range\_NeighDiff\_shell1\_Number & 0.04416 \\ \hline
max\_NeighDiff\_shell1\_NfValence & 0.04376 \\ \hline
mean\_NeighDiff\_shell1\_GSmagmom & 0.0433 \\ \hline
most\_GSmagmom & 0.04239 \\ \hline
range\_NeighDiff\_shell1\_NsUnfilled & 0.03954 \\ \hline
min\_NeighDiff\_shell1\_NsValence & 0.03932 \\ \hline
max\_NeighDiff\_shell1\_AtomicWeight & 0.03905 \\ \hline
max\_NeighDiff\_shell1\_Number & 0.03815 \\ \hline
min\_NfValence & 0.03794 \\ \hline
dev\_NsValence & 0.0373 \\ \hline
maxdiff\_NsUnfilled & 0.03558 \\ \hline
min\_NfUnfilled & 0.03537 \\ \hline
min\_NeighDiff\_shell1\_GSmagmom & 0.03353 \\ \hline
var\_NeighDiff\_shell1\_NsValence & 0.02948 \\ \hline
min\_NpValence & 0.02946 \\ \hline
max\_NsUnfilled & 0.02933 \\ \hline
min\_NeighDiff\_shell1\_GSbandgap & 0.02735 \\ \hline
mean\_GSmagmom & 0.02402 \\ \hline
min\_NpUnfilled & 0.02233 \\ \hline
range\_NeighDiff\_shell1\_NsValence & 0.02171 \\ \hline
min\_NsUnfilled & 0.02051 \\ \hline
min\_GSbandgap & 0.01299 \\ \hline
min\_GSmagmom & 0.00132 \\ \hline
\end{longtable}

\section{Extended Statistics and Visualizations of Materials-Property-Descriptor Database} \label{mpdd:app1}

The three key statistics presented in Figure~\ref{mpdd:fig:dataset} are a small subset of the larger dashboard available at \href{https://phaseslab.org/mpdd}{phaseslab.org/mpdd} web page, which is presented in Figure~\ref{sup:mpdd:dashboard}.

\begin{figure}[H]
    \centering
    \includegraphics[width=0.9\textwidth]{mpdd/mpdddashboard.png}
    \caption{The main MPDD dashboard with statistics over MPDD dataset, as of April 2024, demonstrating the dataset in terms of different levels of "corese-graining" the uniqueness criteria, and coverage of chemical systems of different order.}
    \label{sup:mpdd:dashboard}
\end{figure}

Partial data on all of the MPDD data points can be accessed through a graphical user interface (GUI), available at \href{https://mpdd.org}{mpdd.org}, presented in Figure~\ref{sup:mpdd:gui}. It enables users to query the database based on fields including chemical formula, chemical system, or space group number, to facilitate easy access for users not familiar with programming and those who only need to access a small subset of it.

\begin{figure}[H]
    \centering
    \includegraphics[width=0.9\textwidth]{mpdd/mpddgui.png}
    \caption{A basic MPDD graphical user interface (GUI) set up for easy access to the data based on a couple of common query fields.}
    \label{sup:mpdd:gui}
\end{figure}

The OPTIMADE API \cite{Evans2024DevelopmentsExchange}, as discussed in Section~\ref{mpdd:sec:optimade}, can be accessed using software like \texttt{optimade-python-tools} \cite{Evans2021} or by going to the \href{http://optimade.mpdd.org}{optimade.mpdd.org} endpoint. Figure~\ref{sup:mpdd:optimade} depicts the expected web page, as of May 2024.

\begin{figure}[H]
    \centering
    \includegraphics[width=0.9\textwidth]{mpdd/mpddoptimade.png}
    \caption{Printout of the MPDD OPTIMADE API endpoint page.}
    \label{sup:mpdd:optimade}
\end{figure}

One can use the endpoint page shown in Figure~\ref{sup:mpdd:optimade} to quickly check (1) accessibility of the database, and (2) investigate its schema, by following endpoint like \href{http://optimade.mpdd.org/v1/structures}{optimade.mpdd.org/v1/structures} to see example data response, like the one shown in Figure~\ref{sup:mpdd:optimadeout}, or \href{http://optimade.mpdd.org/v1/info}{optimade.mpdd.org/v1/info} to see MPDD provider information page.

\begin{figure}[H]
    \centering
    \includegraphics[width=0.9\textwidth]{mpdd/mpddoptimadeout.png}
    \caption{An example printout of (partial) MPDD's OPTIMADE response upon a query to the \texttt{http://optimade.mpdd.org/v1/structures} endpoint, depicting human-unreadable but machine-readable output.}
    \label{sup:mpdd:optimadeout}
\end{figure}


\section{Motivation for Multi-Grade Compositional Design of Materials Exemplified with Hf-Zr Powders} \label{nimplex:app1}

As mentioned in Section~\ref{nimplex:ssec:functionallygraded}, different grades of base metals may have very different costs associated with them. For instance, as of December 2023, at \href{https://www.fishersci.com}{Fisher Scientific online store (fishersci.com)}, one can purchase:
\begin{itemize}
    \item High-purity Zr wire: $250cm$ of $0.25mm$-diameter (AA00416CB) for $\$317$ or $\approx 390\frac{\$}{g}$
    \item $99.2\%$ (Zr+$4.5\%$Hf) wire: $200cm$ of $0.25mm$-diameter (AA43334G2) for $\$63$ or $\approx 100\frac{\$}{g}$. 
    \item $99.97\%$ (Hf+$3\%$Zr) wire: $200cm$ of $0.25mm$-diameter (AA10200G2) for $\$200$ or $\approx 156\frac{\$}{g}$. 
\end{itemize} 

Now, if one tries to create FGMs which navigates Zr-rich regions in Hf-containing space, there are two possible choices for Zr source, namely, pure Zr or the (Zr+$4.5\%Hf$) alloy. The first one enables all possible Zr fractions, unlike the latter which establishes the minimum Hf fraction at 4.5\% at the "Zr" corner of the attainable space tetrahedron (anonymous example of this is in Figure~\ref{nimplex:fig:fgmspaces}). Such an ability may be necessary, e.g., to avoid infeasible regions of space, but if not, it represents an unnecessary cost. 

For instance, to obtain (Zr+$10\%wt$Hf) alloy, one can combine high-purity Zr and (Hf+$3\%Zr$) for $\approx 360\frac{\$}{g}$ or equivalently from (Zr+$4.5\%$Hf) and (Hf+$3\%$Zr) for $\approx 103\frac{\$}{g}$, representing 3.5 times cost reduction.

At Fisher Scientific, as of writing this, the pure-Zr wire is only available in $0.25mm$ diameter, thus, the above considerations were restricted to it to keep the comparisons fair. However, for the (Zr+$4.5\%$Hf) grade, many less-expensive form factors are available as it is much more industry relevant. Furthermore, larger package sizes ($\geq50g$) are available driving the cost down further. For instance, the following $1mm$ wires can be purchased:

\begin{itemize}
    \item $99.2\%$ (Zr+$4.5\%$Hf) wire: $10m$ of $1mm$-diameter  (AA14627H2) for $\$130$ or $\approx 2.5\frac{\$}{g}$. 
    \item $99.97\%$ (Hf+$3\%$Zr) wire: $5m$ of $1mm$-diameter (AA10205CC) for $\$580$ or $\approx 11.3\frac{\$}{g}$. 
\end{itemize}

If the above are used, one can now obtain the same (Zr+$10\%wt$Hf) alloy for $\approx 3\frac{\$}{g}$ or 120 times cheaper relative to using high purity Zr in the only available physical form factor.
    

\section{Geometric Cross-Evidence for Factorial Decay of Simplex Space in Equally Dimensional Cartesian Space}  \label{nimplex:app2}

The equation for the fraction of a cube bound by [111] plane, equivalent to the result for $f(4)$ in Section~\ref{nimplex:sec:randomuniformsampling}, can be quickly obtained by considering that the volume of a pyramid is given by $V = \frac{A_B h}{3}$, where $A_B$ is the base area of equilateral triangle $\frac{\sqrt{3}}{4}\times\sqrt{2}^2 = \frac{\sqrt{3}}{2}$ and $h$ is $\frac{1}{\sqrt{3}}$. Thus we get 
$$V = \frac{\frac{\sqrt{3}}{2} \frac{1}{\sqrt{3}}}{3} = \frac{1}{6}$$ 
agreeing with the aforementioned result in Section~\ref{nimplex:sec:randomuniformsampling}.


\section{Bidirectional \texttt{neighborsLink4C} Algorithm found Conceptually}  \label{nimplex:app3}

The equation for "forward" and "backward" jumps in 3-simplex graph corresponding to a quaternary chemical system. 

\begin{minted}[xleftmargin=3\parindent, fontsize=\small]{nim}
proc neighborsLink4C(
    i:int, x:Tensor, neighbors: var seq[seq[int]], ndiv: int): void =
  let jump0 = 1  #binom(x, 0)=1
  let jump1 = binom(1+ndiv-x[0]-x[1], 1)
  let jump2 = binom(2+ndiv-x[0], 2)
  
  if x[0] != 0:
    # quaternary
    neighbors[i].add(i - jump2)
    # quaternary
    neighbors[i].add(i - jump2 - jump1)
    # quaternary
    neighbors[i].add(i - jump2 - jump1 - jump0) 
  
  if x[1] != 0:
    # ternary
    neighbors[i].add(i - jump1)    
    # ternary
    neighbors[i].add(i - jump1 - jump0)         
    # quaternary
    neighbors[i].add(i + jump2 - jump1 - x[1])  
  
  if x[2] != 0:
    # binary
    neighbors[i].add(i - jump0)     
    # ternary
    neighbors[i].add(i + jump1 - jump0)    
    # quaternary
    neighbors[i].add(i + jump2 - jump0 - x[1])  
  
  if x[3] != 0:
    # binary
    neighbors[i].add(i + jump0)        
    # ternary
    neighbors[i].add(i + jump1)          
    # quaternary     
    neighbors[i].add(i + jump2 - x[1])              
\end{minted}

\printbibliography[heading=subbibintoc]